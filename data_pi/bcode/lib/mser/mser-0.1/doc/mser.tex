\documentclass{article} 
\usepackage{visionlab,url}

\newcommand{\rt}{\operatorname{root}}

% ------------------------------------------------------------------------------
\title{An implementation of \\ Maximally Stable Extremal Regions}
\author{Andrea Vedaldi}
% ------------------------------------------------------------------------------

\begin{document}

\maketitle{}

These notes describe the implementation\footnote{The implementation can be downloaded at \url{http://vision.ucla.edu/~vedaldi/code/mser/}}of Maximally Stable Extremal Region of~\cite{matas03robust}. Sect.~\ref{sect:def} formalizes the concept of ``maximally stable extremal regions'' and Sect.~\ref{sect:comp} describes the algorithm that we use to compute them.

% ------------------------------------------------------------------------------
\section{Maximally stable extremal regions}\label{sect:def}
% ------------------------------------------------------------------------------

Here an image $I(x), x\in\Lambda$ is a real function of a discrete domain $\Lambda\subset\real^2$ with a neighborhood topology $\tau$. Elements of $\Lambda$ are called {\em pixels}. Here we consider regular domains (i.e. finite subsets of $\integer^2$) and the elementary 4-way or 8-way neighborhood topology. A {\em level set} $S(x),\ x \in\Lambda$  is the set of pixels that have intensity not greater than $I(x)$, i.e.
\[
 S(x) = \{ y\in\Lambda : I(y) \leq I(x) \}.
\]
A {\em path} $(x_1,\dots,x_n)$ is a sequence of pixels such that $x_i$ and $x_{i+1}$ are neighbors for all $i=1,\dots,n-1$. A {\em connected component} $C$ of the domain $\Lambda$ is a subset $C\subset\Lambda$ for which each pair $(x_1,x_2)\in C^2$ of pixels is connected by a path fully contained in $C$. The connected component is {\em maximal} if any other connected component $C'$ containing $C$ is equal to $C$. An {\em extremal region} $R$ is a maximal connected component of a level set $S(x)$.

Among all extremal regions $\mathcal{R}(I)$ of image $I$, we are interested to the onesi that satisfy certain stability criterion which we introduce next. Let the {\em level} $I(R)$ of the extremal region $R$ be the maximum image intensity $\max_{x\in R} I(x)$ attained in the region $R$. Consider the biggest extremal region $R_{+\Delta}$ contained in $R$ that has intensity greater at least $\Delta\geq 0$ than $R$, i.e.
\be\label{eq:rplus}
 R_{+\Delta} = \argmax \{ |Q| : Q \in \mathcal{R}(I),\ Q\subset R,\ I(Q)\geq I(R)+\Delta \}.
\ee
Similarly, consider the smallest extremal region containing $R$ that has intensity at least $\Delta$ smaller than $R$, i.e.
\be\label{eq:rminus}
 R_{-\Delta} = \argmin \{ |Q| : Q \in \mathcal{R}(I),\ Q\supset R,\ I(Q)\leq I(R)-\Delta \}.
\ee
Consider the area variation
\[
   \rho(R;\Delta) = \frac{|R_{+\Delta}| - |R_{-\Delta}|}{|R|}.
\]
The region $R$ is {\em maximally stable} if it is a minimum for the area variation, i.e. if $\rho(R;\Delta)$ is smaller than $\rho(Q;\Delta)$ for any region $Q$ ``immediately contained'' or ``immediately containing'' $R$. We say that an extremal region $R$ {\em immediately contains} another extremal region $Q$ if $R\supset Q$ and if $R'$ is another extremal region with $R \supset R' \supset Q$, then $R'=R$.

% ------------------------------------------------------------------------------
\section{Regions computation}\label{sect:comp}
% ------------------------------------------------------------------------------

We describe an efficient algorithm for the computation of the maximally stable extremal regions of an image $I(x)$ defined on a discrete domain $\Lambda$. 

% ------------------------------------------------------------------------------
\subsection{Enumerating extremal regions}
% ------------------------------------------------------------------------------

We describe first a method to enumerate all extremal regions of a given image $I$. Let $x_1,x_2,\ldots,x_N \in \Lambda$ be a sorting of the image pixels by increasing inteisty value, i.e.
\[ I(x_1) \leq I(x_2) \leq \dots I(x_N). \]
We compute extremal regions incrementally, by considering larger and larger image subdomains $\Lambda_t = \{x_1,x_2,\dots,x_t\}\subset \Lambda$ for $t=1,\dots,N$. Denote by $I_t = I|_{\Lambda_t}$ the restriction of the image $I$ to the subset $\Lambda_t$.

For $t=1$, $\Lambda_1 = \{x_1\}$ is trivially an extremal region of the image $I_!$ and level $I(x_1)$. For $t=2$, either $x_1$ and $x_2$ are connected and $\Lambda_2$ is an extremal region of $I_2$, or they are not and $\{x_2\}$ is an extremal region of $\Lambda_2$. Moreover $\Lambda_1$ is an extremal region of $I_2$ if, and only if, $I(x_2)\not=I(x_1)$. This is captured in general by:

\begin{lemma}\label{lemma:er-fund} 
Let $t$ be one of $1,2,\dots,N-1$. Let $R_1,\ldots,R_K$ be all the extremal regions of $I_t$. Let
\begin{itemize}
\item
$\mathcal{K}_1$ the subset of indices $k$ for which $I(R_k) \not= I(x_{t+1})$ and 
\item
$\mathcal{K}_2$ the subset of indices $k$ for which $I(R_k) = I(x_{t+1})$ but $x_{t+1}$ is {\em not} connected to $R_k$ and
\item
let $\mathcal{K}_3$ be the subset of indices $k$ for which $x_{t+1}$ is connected to $R_k$. 
\end{itemize}
Then
	\begin{enumerate}
		\item for all $k\in\mathcal{K}_1 \cup \mathcal{K}_2$ the set $R_k$ is an extremal region of $I_{t+1}$;
		
		\item the set $R = \{x_{t+1}\} \cup_{k \in\mathcal{K}_3}$ is an extremal region of $I_{t+1}$;
		
		\item all extremal regions of $I_{t+1}$ are obtained either as (1) or (2). 
	\end{enumerate}
\end{lemma}

\begin{proof}
By definition each $R_k$ is a maximal connected component of the set $S_t(R_k) = \{x\in\Lambda_t : I(x) \leq I(R_k)\}$. If $k\in\mathcal{K}_1$, then $I(R_k)\not= I(x_{t+1})$, $S_t(R_k) = S_{t+1}(R_k)$ and $R_k$ is a maximal connected component of $S_{t+1}(R_k)$ as well. If $k\not\in\mathcal{K}_1$, then $S_{t+1}(R_k) = S(R_k) \cup \{x_{t+1}\}$. However if $k\in\mathcal{K}_2$, then $R_k$ and $x_{t+1}$ are not neighbors and $R_k$ is still maximal in $S_{t+1}(R_k)$. Finally, $\{x_{t+1}\}$ together with all the regions $R_k$ of level $I(R_k) \leq I(x_{t+1})$ which are neighbors of $x_{t+1}$, i.e. $k\in\mathcal{K}_3$, constitute a new extremal region. To see this, note that (i) $R\subset S(x_{t+1})$, (ii) $R$ is connected because the subregions $R_k$ are connected and any two points in two different subregions are connected through $x_{t+1}$ by construction and (iii) $R$ is maximal as if not, one could add a pixel $y \in \Lambda_t = \Lambda_{t+1} - \{x_{t+1}\}$ to $R$ that would be either an extension of one of the extremal regions $R_k$ of image $I_t$ or $\{y\}$ would be a new extremal region of image $I_t$ by itself.
	
Finally, we need to show that the listing is exhaustive. So let $R$ be an extremal region of image $I_{t+1}$. If $R \subset S_t(R)$, then $x_{t+1}\not\in R$ and $R$ is equal to some $R_k$ for $k\in\mathcal{K}_1\cup\mathcal{K}_2$ by the inductive hypotesis. If, on the other hand, $x_{t+1}\in R$, then R is obtained as (2).
\end{proof}

Lemma~\ref{lemma:er-fund} suggests a simple algorithm to enumerate extremal regions. The idea is to consider one pixel at time in the order $x_1, x_2,\dots$ growing extremal regions for the intermediate images $I_t$ until $I_N=I$ is reached.

Formally, this process can be implemented by means of a forest of pixels. At time $t$ the forest represents all the union operations that have been performed so far according to point (2) of Lemma~\ref{lemma:er-fund}. Since extremal regions are only generated by such union operations, the tree stores all the extremal regions of all intermediate images $I_1,\dots,I_t$. 

Let us consider the addition of pixel $x_{t+1}$ to the forest. Following Lemma~\ref{lemma:er-fund}, we must search for all extremal regions $R_1,\dots,R_k$ of image $I_t$ which are neighbors of $x_{t+1}$ and join them to $x_{t+1}$ to obtain the new region $R$. This is done by scanning the neighbors $y\in\Lambda_{t}$ of $x_{t+1}$ and, for each of them, climbing the tree in search for the appropriate extremal regions $R_k$. In practice, we simply take the union of all sets $S(y) \cup S(\pi(y)) \cup S(\pi^2(y)) \cup \dots \cup S(\rt(y)) = S(\rt(y))$, where $S(y)$ is the subtree rooted at $y$, $\pi(y)$ is the parent of $y$ and $\rt(y)$ is the root of the tree that contains $y$. While only some of $S(\pi^n(y))$ are indeed extremal regions of image $I_t$, $S(\rt(y))$ always is and, since it covers all other subsets anyway, it is sufficient to join that.  The join operation is then encoded in the forset by making $x_{t+1}$ parent of $\rt(y)$, i.e. $\pi(\rt(y)) \leftarrow x_{t+1}$.

This basic algorithm can be improved significantly by keeping the tree balanced. This is an optimization of the join operation, for which $x_{t+1}$ is not necessarily added to the forest as root; instead one uses as root one of the nodes $\rt(y)$ with the goal keeping the tree height short. Although this disrupts partially the property of the forest (some of the extremal regions of the intermediate images $I_1, I_2, \ldots$ are lost), the relevant information (i.e. the regions that are extremal regions of $I$) is preserved, as it can be verified. In particular, regions can be emitted as soon as condition (1) of the Lemma is encountered, which correspond to the case $I(y) \not= I(\pi(y))$.

% ------------------------------------------------------------------------------
\subsection{Computing the stability score}
% ------------------------------------------------------------------------------

Once the extremal region tree is computed, we need to calculate the area variation for each region and then selecting the maximally stable one.
The area $|R|$ of each region is computed efficiently as explained in Sect.~\ref{sect:er.int}. In order to compute the area variation of a region $R$, we need to figure out the regions $R_{-\Delta}$ and $R_{+\Delta}$. 
To do this we begin by arranging the extremal regions of image $I$into a tree where $R$ is parent of $R'$ if $R$ immediately contains $R'$. Then each region $R$ is considered and the tree is explored to find a region $Q$ for which $R=Q_{-\Delta}$ and the region $R_{+\Delta}$. This is done by scanning the regions $R_0=R$, $R_1=\pi(R_0),$ $R_2=\pi(R_1)$ and so on. If a region $Q=R_i$ satisfies $Q_{-\Delta}=R_0$, then
\[
   I(R_0) \leq I(R_i) - \Delta < I(R_1).
\]
The condition is not necessary though; according to \eqref{eq:rminus} we need to keep the region of maximum area among all the candidate ones.
Similarly, if $R_i=R_{+\Delta}$, then
\[
   I(R_i) \leq I(R_0) + \Delta < I(R_{i+1}).
\]
In this case the condition is also sufficient as at most one of such regions exist.

% ------------------------------------------------------------------------------
\subsection{Cleaning up}\label{sect:er.cleanup}
% ------------------------------------------------------------------------------

The stability score alone may not be sufficient to select only useful regions. In the cleanup phase we
\begin{itemize}
\item remove very small and very big regions;
\item remove regions which have too high area variation (even if they are indeed minima of the variation score);
\item remove duplicated regions.
\end{itemize}
Duplicated regions arise because, due to noise, the same mode of the local minima score may correspond to more than one local minimum. Duplicated regions are easily found by comparing each MSER $R$ with the MSER $R'$ immediately containing $R$ and removing $R$ if they are too similar.

% ------------------------------------------------------------------------------
\subsection{Fitting elliptical regions}\label{sect:er.int}
% ------------------------------------------------------------------------------

Fitting elliptical regions amount to computing for each maximally stable extremal region $R$ the first and second order moments, i.e.
\[
 \mu(R) = \frac{1}{|R|} \sum_{x\in R} x,
 \qquad
 \Sigma(R) = \frac{1}{|R|} \sum_{x\in R} (x-\mu)(x-\mu)^\top.
\]
Rather than considering directly the centered moment $\Sigma(R)$, it is computationally more convenient to compute
\[
  M(R) = \frac{1}{R} \sum_{x\in R}xx^\top
\]
and use the fact that $\Sigma(R) = M(R) - \mu(R)\mu(R)^\top$. The advantage is that any quantity which is obtained by integrating a function $f(x),$ $x\in\Lambda$ of the image domain (in particular $f(x)=x$ and $f(x)=xx^\top$) can be computed for all regions at once by visiting (in breath first order and from the leaves) each pixel of the forest and summing its value to the parent.


% ------------------------------------------------------------------------------
\bibliographystyle{plain}
\bibliography{mser}
% ------------------------------------------------------------------------------

\end{document}
